\documentclass[11pt,a4paper,twocolumn]{article}
\usepackage[utf8]{inputenc}
\usepackage{amsmath}
\usepackage{graphicx}
\usepackage{hyperref}
\usepackage{setspace}

\title{Assignment5-Probability and Random variables}
\author{Aravind A Anil}
\date{\today}

\begin{document}

\maketitle

\textbf{Problem-Statement:}A die is thrown again and again until three sixes are obtained.Find the probability of obtaining third six in the sixth throw of the die
\\\textbf{Solution:}Here we need to find the probability of obtaining third six in the sixth throw of die
\\P(getting 3rd six in 6th throw)
\begin{align*}
\\&=\text{P(getting 2 sixes in 5 throws)}\\
&\quad\times\text{P(getting head in sixth throw)}
\\&=\text{P(getting 2 sixes in 5 throws)}\times \frac{1}{6}
\end{align*}
\textbf{Calculating P(getting 2 sixes in 5 throws)}
\begin{table}[h!]
    \centering
    \begin{tabular}{|c|c|}
    \hline
        variables & Description\\
        \hline
        X& number of sixes in 5 throws\\
        \hline
        p&probability of getting six\\
        \hline
        q&probability of not getting six\\
        \hline
        n&number of times die is thrown\\
        \hline
        \end{tabular}
        \label{tab:my_label}
\end{table}
\\n=5, $\text{p}=\frac{1}{6}$ and $\text{q}=\frac{5}{6}$
\\Here checking whether six is obtained or not in a given throw is a 
Bernoulli's trail.
\\Since we are throwing the die 5 times 
\\X has a Binomial Distribution
\\P(X=x)=$^{n}C_xp^{x}q^{n-x}$
\\We need to find P(getting 2 sixes in 5 throws)
\begin{align*}
    P(X=2)&=^{5}C_2\Big(\frac{1}{6}\Big)^{2}\Big(\frac{5}{6}\Big)^{3}\\&=\frac{5\times4}{2}\times\Big(\frac{1}{6}\Big)^{2}\times\Big(\frac{5}{6}\Big)^{3}
    \\&=10\times5^3\times\Big(\frac{1}{6}\Big)^{5}\\
    &=\frac{10\times5^{3}}{6^{5}}
\end{align*}
Hence the required probability is,
\begin{align*}
    &=P(X=2)\times\frac{1}{6}\\
    &=\frac{10\times5^{3}}{6^{6}}\times\frac{1}{6}\\
    &=\frac{10\times5^{3}}{6^{6}}\\
    &=\frac{10\times125}{46656}\\
    &=\frac{625}{23328}\\
    &=.0267
\end{align*}




\end{document}
